% Options for packages loaded elsewhere
\PassOptionsToPackage{unicode}{hyperref}
\PassOptionsToPackage{hyphens}{url}
%
\documentclass[
]{book}
\usepackage{amsmath,amssymb}
\usepackage{lmodern}
\usepackage{iftex}
\ifPDFTeX
  \usepackage[T1]{fontenc}
  \usepackage[utf8]{inputenc}
  \usepackage{textcomp} % provide euro and other symbols
\else % if luatex or xetex
  \usepackage{unicode-math}
  \defaultfontfeatures{Scale=MatchLowercase}
  \defaultfontfeatures[\rmfamily]{Ligatures=TeX,Scale=1}
\fi
% Use upquote if available, for straight quotes in verbatim environments
\IfFileExists{upquote.sty}{\usepackage{upquote}}{}
\IfFileExists{microtype.sty}{% use microtype if available
  \usepackage[]{microtype}
  \UseMicrotypeSet[protrusion]{basicmath} % disable protrusion for tt fonts
}{}
\makeatletter
\@ifundefined{KOMAClassName}{% if non-KOMA class
  \IfFileExists{parskip.sty}{%
    \usepackage{parskip}
  }{% else
    \setlength{\parindent}{0pt}
    \setlength{\parskip}{6pt plus 2pt minus 1pt}}
}{% if KOMA class
  \KOMAoptions{parskip=half}}
\makeatother
\usepackage{xcolor}
\usepackage{color}
\usepackage{fancyvrb}
\newcommand{\VerbBar}{|}
\newcommand{\VERB}{\Verb[commandchars=\\\{\}]}
\DefineVerbatimEnvironment{Highlighting}{Verbatim}{commandchars=\\\{\}}
% Add ',fontsize=\small' for more characters per line
\usepackage{framed}
\definecolor{shadecolor}{RGB}{248,248,248}
\newenvironment{Shaded}{\begin{snugshade}}{\end{snugshade}}
\newcommand{\AlertTok}[1]{\textcolor[rgb]{0.94,0.16,0.16}{#1}}
\newcommand{\AnnotationTok}[1]{\textcolor[rgb]{0.56,0.35,0.01}{\textbf{\textit{#1}}}}
\newcommand{\AttributeTok}[1]{\textcolor[rgb]{0.77,0.63,0.00}{#1}}
\newcommand{\BaseNTok}[1]{\textcolor[rgb]{0.00,0.00,0.81}{#1}}
\newcommand{\BuiltInTok}[1]{#1}
\newcommand{\CharTok}[1]{\textcolor[rgb]{0.31,0.60,0.02}{#1}}
\newcommand{\CommentTok}[1]{\textcolor[rgb]{0.56,0.35,0.01}{\textit{#1}}}
\newcommand{\CommentVarTok}[1]{\textcolor[rgb]{0.56,0.35,0.01}{\textbf{\textit{#1}}}}
\newcommand{\ConstantTok}[1]{\textcolor[rgb]{0.00,0.00,0.00}{#1}}
\newcommand{\ControlFlowTok}[1]{\textcolor[rgb]{0.13,0.29,0.53}{\textbf{#1}}}
\newcommand{\DataTypeTok}[1]{\textcolor[rgb]{0.13,0.29,0.53}{#1}}
\newcommand{\DecValTok}[1]{\textcolor[rgb]{0.00,0.00,0.81}{#1}}
\newcommand{\DocumentationTok}[1]{\textcolor[rgb]{0.56,0.35,0.01}{\textbf{\textit{#1}}}}
\newcommand{\ErrorTok}[1]{\textcolor[rgb]{0.64,0.00,0.00}{\textbf{#1}}}
\newcommand{\ExtensionTok}[1]{#1}
\newcommand{\FloatTok}[1]{\textcolor[rgb]{0.00,0.00,0.81}{#1}}
\newcommand{\FunctionTok}[1]{\textcolor[rgb]{0.00,0.00,0.00}{#1}}
\newcommand{\ImportTok}[1]{#1}
\newcommand{\InformationTok}[1]{\textcolor[rgb]{0.56,0.35,0.01}{\textbf{\textit{#1}}}}
\newcommand{\KeywordTok}[1]{\textcolor[rgb]{0.13,0.29,0.53}{\textbf{#1}}}
\newcommand{\NormalTok}[1]{#1}
\newcommand{\OperatorTok}[1]{\textcolor[rgb]{0.81,0.36,0.00}{\textbf{#1}}}
\newcommand{\OtherTok}[1]{\textcolor[rgb]{0.56,0.35,0.01}{#1}}
\newcommand{\PreprocessorTok}[1]{\textcolor[rgb]{0.56,0.35,0.01}{\textit{#1}}}
\newcommand{\RegionMarkerTok}[1]{#1}
\newcommand{\SpecialCharTok}[1]{\textcolor[rgb]{0.00,0.00,0.00}{#1}}
\newcommand{\SpecialStringTok}[1]{\textcolor[rgb]{0.31,0.60,0.02}{#1}}
\newcommand{\StringTok}[1]{\textcolor[rgb]{0.31,0.60,0.02}{#1}}
\newcommand{\VariableTok}[1]{\textcolor[rgb]{0.00,0.00,0.00}{#1}}
\newcommand{\VerbatimStringTok}[1]{\textcolor[rgb]{0.31,0.60,0.02}{#1}}
\newcommand{\WarningTok}[1]{\textcolor[rgb]{0.56,0.35,0.01}{\textbf{\textit{#1}}}}
\usepackage{longtable,booktabs,array}
\usepackage{calc} % for calculating minipage widths
% Correct order of tables after \paragraph or \subparagraph
\usepackage{etoolbox}
\makeatletter
\patchcmd\longtable{\par}{\if@noskipsec\mbox{}\fi\par}{}{}
\makeatother
% Allow footnotes in longtable head/foot
\IfFileExists{footnotehyper.sty}{\usepackage{footnotehyper}}{\usepackage{footnote}}
\makesavenoteenv{longtable}
\usepackage{graphicx}
\makeatletter
\def\maxwidth{\ifdim\Gin@nat@width>\linewidth\linewidth\else\Gin@nat@width\fi}
\def\maxheight{\ifdim\Gin@nat@height>\textheight\textheight\else\Gin@nat@height\fi}
\makeatother
% Scale images if necessary, so that they will not overflow the page
% margins by default, and it is still possible to overwrite the defaults
% using explicit options in \includegraphics[width, height, ...]{}
\setkeys{Gin}{width=\maxwidth,height=\maxheight,keepaspectratio}
% Set default figure placement to htbp
\makeatletter
\def\fps@figure{htbp}
\makeatother
\setlength{\emergencystretch}{3em} % prevent overfull lines
\providecommand{\tightlist}{%
  \setlength{\itemsep}{0pt}\setlength{\parskip}{0pt}}
\setcounter{secnumdepth}{5}
\usepackage{booktabs}
\usepackage{amsthm}
\makeatletter
\def\thm@space@setup{%
  \thm@preskip=8pt plus 2pt minus 4pt
  \thm@postskip=\thm@preskip
}
\makeatother
\ifLuaTeX
  \usepackage{selnolig}  % disable illegal ligatures
\fi
\usepackage[]{natbib}
\bibliographystyle{apalike}
\IfFileExists{bookmark.sty}{\usepackage{bookmark}}{\usepackage{hyperref}}
\IfFileExists{xurl.sty}{\usepackage{xurl}}{} % add URL line breaks if available
\urlstyle{same} % disable monospaced font for URLs
\hypersetup{
  pdftitle={Data Science For Business},
  pdfauthor={Eddwin Cheteni},
  hidelinks,
  pdfcreator={LaTeX via pandoc}}

\title{Data Science For Business}
\author{Eddwin Cheteni}
\date{2022-10-28}

\begin{document}
\maketitle

{
\setcounter{tocdepth}{1}
\tableofcontents
}
\begin{figure}
\centering
\includegraphics{images/ds4business.png}
\caption{cover-image}
\end{figure}

\hypertarget{preface}{%
\chapter{Preface}\label{preface}}

This e-book is intended to be used for Practical Data Science using either \textbf{R} or \textbf{Python} language.

The book is organised in such a way that each section expands the use-cases as applied to different fields. Because of this, the sections are not inclusive, i.e.~knowledge of the fundamentals will be required in most cases going forward.

\hypertarget{intro}{%
\chapter{Introduction}\label{intro}}

This book will focuses on project-based learning of Data Science as applied to various micro-economic data:

\begin{itemize}
\tightlist
\item
  Industry-specific data - marketing and sales
\end{itemize}

The main idea is to introduce corporate sectors (motor, retail, logistic industry) to best practice approaches to leverage their business operations through use of Data Science. All the illustrations will be using case studies to make you understand how best can you implement Data Science to solve business problems. The table below shows various steps as applied to different companies.\citep{eddie2022}

\hypertarget{import-data-extractload}{%
\section{Import data (extract,load)}\label{import-data-extractload}}

This section deals with extraction of data from various sources depending on the use-case.

\hypertarget{data-wrangling-transform}{%
\section{Data wrangling (transform)}\label{data-wrangling-transform}}

This outline steps to uncover useful insights from the data, thus including transformation,

\hypertarget{model-selection-modeling}{%
\section{Model selection (modeling)}\label{model-selection-modeling}}

Various approaches are used to build machine learning models based on the use-cases.

\hypertarget{evaluation}{%
\section{Evaluation}\label{evaluation}}

Different machine learning models have different ways to evaluate their performance.

\hypertarget{deployment}{%
\section{Deployment}\label{deployment}}

For machine learning models to be useful, one needs to put them into production so as to add value to various businesses.

\hypertarget{tidy}{%
\chapter{Data Wrangling}\label{tidy}}

\hypertarget{overview-mexico-toy-sales-project}{%
\section{Overview (Mexico Toy Sales project)}\label{overview-mexico-toy-sales-project}}

Sales \& inventory data for a fictitious chain of toy stores in Mexico called Maven Toys, including information about products, stores, daily sales transactions, and current inventory levels at each location.

In this project, you'll work with a dataset with 830,940 records and 14 fields for sales \& inventory data for a fictitious chain of toy stores in Mexico called Maven Toys, including information about products, stores, daily sales transactions, and current inventory levels at each location pulled from \includegraphics{https://maven-datasets.s3.amazonaws.com/Maven+Toys/Maven+Toys+Data.zip}

Some of the things you'll learn in this project are:

\begin{itemize}
\item
  How to organize information using basic Python data structures.
\item
  How to import data from CSV files and clean it using the pandas library.
\item
  How to create data visualizations like scatter and box plots.
\item
  How to examine the relationship between two variables using correlation.
\end{itemize}

\hypertarget{recommended-analysis}{%
\subsection{Recommended Analysis}\label{recommended-analysis}}

\begin{enumerate}
\def\labelenumi{\arabic{enumi}.}
\item
  Which product categories drive the biggest profits?
\item
  Is this the same across store locations?
\item
  Can you find any seasonal trends or patterns in the sales data?
\item
  Are sales being lost with out-of-stock products at certain locations?
\end{enumerate}

How much money is tied up in inventory at the toy stores?
How long will it last?

\hypertarget{import-data}{%
\section{Import data}\label{import-data}}

There are so many packages in \textbf{R} that can be used to extract data into Rstudio environment.
For \textbf{Mexico Toy Sales} data, we will use \texttt{tidyverse} package.

\begin{Shaded}
\begin{Highlighting}[]
\FunctionTok{library}\NormalTok{(tidyverse) }\CommentTok{\# core for importing data}
\end{Highlighting}
\end{Shaded}

\begin{verbatim}
## -- Attaching packages --------------------------------------- tidyverse 1.3.2 --
## v ggplot2 3.3.6     v purrr   0.3.4
## v tibble  3.1.7     v dplyr   1.0.9
## v tidyr   1.2.0     v stringr 1.4.0
## v readr   2.1.2     v forcats 0.5.1
## -- Conflicts ------------------------------------------ tidyverse_conflicts() --
## x dplyr::filter() masks stats::filter()
## x dplyr::lag()    masks stats::lag()
\end{verbatim}

\begin{Shaded}
\begin{Highlighting}[]
\NormalTok{temp }\OtherTok{\textless{}{-}} \FunctionTok{tempfile}\NormalTok{()}
\FunctionTok{download.file}\NormalTok{(}\StringTok{"https://maven{-}datasets.s3.amazonaws.com/Maven+Toys/Maven+Toys+Data.zip"}\NormalTok{,temp, }\AttributeTok{model=}\StringTok{"wb"}\NormalTok{)}
\end{Highlighting}
\end{Shaded}

\hypertarget{data-dictionary}{%
\subsection{Data dictionary}\label{data-dictionary}}

\begin{longtable}[]{@{}
  >{\raggedright\arraybackslash}p{(\columnwidth - 2\tabcolsep) * \real{0.2432}}
  >{\raggedright\arraybackslash}p{(\columnwidth - 2\tabcolsep) * \real{0.7568}}@{}}
\toprule()
\begin{minipage}[b]{\linewidth}\raggedright
Field
\end{minipage} & \begin{minipage}[b]{\linewidth}\raggedright
Description
\end{minipage} \\
\midrule()
\endhead
Store\_ID & Store ID \\
Product\_ID & Product ID \\
Stock\_On\_Hand & Stock quantity of the product in the store (inventory) \\
Store\_Name & Store name \\
Store\_City & City in Mexico where the store is located \\
Store\_Location & Location in the city where the store is located \\
Store\_Open\_Date & Date when the store was opened \\
Sale\_ID & Sale ID \\
Units & Units sold \\
Product\_ID & Product ID \\
Product\_Name & Product name \\
Product\_Category & Product Category \\
Product\_Cost & Product cost (\$USD) \\
Product\_Price & Product retail price (\$USD) \\
\bottomrule()
\end{longtable}

\begin{Shaded}
\begin{Highlighting}[]
\CommentTok{\# the data is in comma{-}delimited, so we will use \textasciigrave{}read\_csv\textasciigrave{} from \textasciigrave{}tidyverse\textasciigrave{} package}
\CommentTok{\# inventory data}
\NormalTok{inventory }\OtherTok{\textless{}{-}} \FunctionTok{read\_csv}\NormalTok{(}\FunctionTok{unz}\NormalTok{(temp, }\StringTok{"inventory.csv"}\NormalTok{))}
\end{Highlighting}
\end{Shaded}

\begin{verbatim}
## Rows: 1593 Columns: 3
## -- Column specification --------------------------------------------------------
## Delimiter: ","
## dbl (3): Store_ID, Product_ID, Stock_On_Hand
## 
## i Use `spec()` to retrieve the full column specification for this data.
## i Specify the column types or set `show_col_types = FALSE` to quiet this message.
\end{verbatim}

\begin{Shaded}
\begin{Highlighting}[]
\NormalTok{inventory}\SpecialCharTok{\%\textgreater{}\%}\FunctionTok{head}\NormalTok{()}
\end{Highlighting}
\end{Shaded}

\begin{verbatim}
## # A tibble: 6 x 3
##   Store_ID Product_ID Stock_On_Hand
##      <dbl>      <dbl>         <dbl>
## 1        1          1            27
## 2        1          2             0
## 3        1          3            32
## 4        1          4             6
## 5        1          5             0
## 6        1          6            79
\end{verbatim}

\begin{Shaded}
\begin{Highlighting}[]
\CommentTok{\# the data is in comma{-}delimited, so we will use \textasciigrave{}read\_csv\textasciigrave{} from \textasciigrave{}tidyverse\textasciigrave{} package}
\CommentTok{\# inventory data}
\NormalTok{products }\OtherTok{\textless{}{-}} \FunctionTok{read\_csv}\NormalTok{(}\FunctionTok{unz}\NormalTok{(temp, }\StringTok{"products.csv"}\NormalTok{))}
\end{Highlighting}
\end{Shaded}

\begin{verbatim}
## Rows: 35 Columns: 5
## -- Column specification --------------------------------------------------------
## Delimiter: ","
## chr (4): Product_Name, Product_Category, Product_Cost, Product_Price
## dbl (1): Product_ID
## 
## i Use `spec()` to retrieve the full column specification for this data.
## i Specify the column types or set `show_col_types = FALSE` to quiet this message.
\end{verbatim}

\begin{Shaded}
\begin{Highlighting}[]
\NormalTok{products}\SpecialCharTok{\%\textgreater{}\%}\FunctionTok{head}\NormalTok{()}
\end{Highlighting}
\end{Shaded}

\begin{verbatim}
## # A tibble: 6 x 5
##   Product_ID Product_Name     Product_Category Product_Cost Product_Price
##        <dbl> <chr>            <chr>            <chr>        <chr>        
## 1          1 Action Figure    Toys             $9.99        $15.99       
## 2          2 Animal Figures   Toys             $9.99        $12.99       
## 3          3 Barrel O' Slime  Art & Crafts     $1.99        $3.99        
## 4          4 Chutes & Ladders Games            $9.99        $12.99       
## 5          5 Classic Dominoes Games            $7.99        $9.99        
## 6          6 Colorbuds        Electronics      $6.99        $14.99
\end{verbatim}

\begin{Shaded}
\begin{Highlighting}[]
\CommentTok{\# the data is in comma{-}delimited, so we will use \textasciigrave{}read\_csv\textasciigrave{} from \textasciigrave{}tidyverse\textasciigrave{} package}
\CommentTok{\# inventory data}
\NormalTok{sales }\OtherTok{\textless{}{-}} \FunctionTok{read\_csv}\NormalTok{(}\FunctionTok{unz}\NormalTok{(temp, }\StringTok{"sales.csv"}\NormalTok{))}
\end{Highlighting}
\end{Shaded}

\begin{verbatim}
## Rows: 829262 Columns: 5
## -- Column specification --------------------------------------------------------
## Delimiter: ","
## dbl  (4): Sale_ID, Store_ID, Product_ID, Units
## date (1): Date
## 
## i Use `spec()` to retrieve the full column specification for this data.
## i Specify the column types or set `show_col_types = FALSE` to quiet this message.
\end{verbatim}

\begin{Shaded}
\begin{Highlighting}[]
\NormalTok{sales}\SpecialCharTok{\%\textgreater{}\%}\FunctionTok{head}\NormalTok{()}
\end{Highlighting}
\end{Shaded}

\begin{verbatim}
## # A tibble: 6 x 5
##   Sale_ID Date       Store_ID Product_ID Units
##     <dbl> <date>        <dbl>      <dbl> <dbl>
## 1       1 2017-01-01       24          4     1
## 2       2 2017-01-01       28          1     1
## 3       3 2017-01-01        6          8     1
## 4       4 2017-01-01       48          7     1
## 5       5 2017-01-01       44         18     1
## 6       6 2017-01-01        1         31     1
\end{verbatim}

\begin{Shaded}
\begin{Highlighting}[]
\CommentTok{\# the data is in comma{-}delimited, so we will use \textasciigrave{}read\_csv\textasciigrave{} from \textasciigrave{}tidyverse\textasciigrave{} package}
\CommentTok{\# inventory data}
\NormalTok{stores }\OtherTok{\textless{}{-}} \FunctionTok{read\_csv}\NormalTok{(}\FunctionTok{unz}\NormalTok{(temp, }\StringTok{"stores.csv"}\NormalTok{))}
\end{Highlighting}
\end{Shaded}

\begin{verbatim}
## Rows: 50 Columns: 5
## -- Column specification --------------------------------------------------------
## Delimiter: ","
## chr  (3): Store_Name, Store_City, Store_Location
## dbl  (1): Store_ID
## date (1): Store_Open_Date
## 
## i Use `spec()` to retrieve the full column specification for this data.
## i Specify the column types or set `show_col_types = FALSE` to quiet this message.
\end{verbatim}

\begin{Shaded}
\begin{Highlighting}[]
\NormalTok{stores}\SpecialCharTok{\%\textgreater{}\%}\FunctionTok{head}\NormalTok{()}
\end{Highlighting}
\end{Shaded}

\begin{verbatim}
## # A tibble: 6 x 5
##   Store_ID Store_Name               Store_City  Store_Location Store_Open_Date
##      <dbl> <chr>                    <chr>       <chr>          <date>         
## 1        1 Maven Toys Guadalajara 1 Guadalajara Residential    1992-09-18     
## 2        2 Maven Toys Monterrey 1   Monterrey   Residential    1995-04-27     
## 3        3 Maven Toys Guadalajara 2 Guadalajara Commercial     1999-12-27     
## 4        4 Maven Toys Saltillo 1    Saltillo    Downtown       2000-01-01     
## 5        5 Maven Toys La Paz 1      La Paz      Downtown       2001-05-31     
## 6        6 Maven Toys Mexicali 1    Mexicali    Commercial     2003-12-13
\end{verbatim}

Now we need to join our data from different sources into one table.
We will use \texttt{left\_join} to achieve that.

\hypertarget{methods}{%
\chapter{Methods}\label{methods}}

We describe our methods in this chapter.

Math can be added in body using usual syntax like this

\hypertarget{math-example}{%
\section{math example}\label{math-example}}

\(p\) is unknown but expected to be around 1/3. Standard error will be approximated

\[
SE = \sqrt(\frac{p(1-p)}{n}) \approx \sqrt{\frac{1/3 (1 - 1/3)} {300}} = 0.027
\]

You can also use math in footnotes like this\footnote{where we mention \(p = \frac{a}{b}\)}.

We will approximate standard error to 0.027\footnote{\(p\) is unknown but expected to be around 1/3. Standard error will be approximated

  \[
  SE = \sqrt(\frac{p(1-p)}{n}) \approx \sqrt{\frac{1/3 (1 - 1/3)} {300}} = 0.027
  \]}

\hypertarget{applications}{%
\chapter{Applications}\label{applications}}

Some \emph{significant} applications are demonstrated in this chapter.

\hypertarget{example-one}{%
\section{Example one}\label{example-one}}

\hypertarget{example-two}{%
\section{Example two}\label{example-two}}

\hypertarget{final-words}{%
\chapter{Final Words}\label{final-words}}

We have finished a nice book.

  \bibliography{book.bib,packages.bib}

\end{document}
